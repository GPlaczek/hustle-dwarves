\documentclass{article}

\usepackage{polski}
\usepackage{algorithm}
\usepackage{algpseudocode}
\usepackage{multicol}

\usepackage[margin=1.0in]{geometry}

\author{Grzegorz Płaczek (148071), Łukasz Kania (148077)}
\title{Opis algorytmu rozwiązującego problem rozproszonej sekcji krytycznej o rozmiarze N}

\begin{document}
    \maketitle

    \section{Algorytm}

    W przypadku obu sekcji krytycznych zdecydowaliśmy się wybrać odpowiednio dostosowany algorytm Ricarta-Agrawali.

    \subsection{Przydział fuchy do krasnala}

    \begin{enumerate}
        \item Krasnal, który chce ubiegać się o fuchę wysyła \texttt{REQ\_JOB} wraz z wartością zegara Lamporta do wszystkich pozostałych krasnali i czeka na odpowiedzi.
        Robi to niezależnie od tego, czy dostępne dostępne są jakiekolwiek fuchy.
        \item Krasnal kończy oczekiwanie po otrzymaniu \texttt{K-N} (\texttt{K} to liczba krasnali \texttt{N} to liczba dostępnych fuch) wiadomości \texttt{ACK\_JOB}.
        Liczba \texttt{N} może się zmieniać w trakcie oczekiwania.
        \item W czasie oczekiwania, krasnal odpowiednio zwiększa swój lokalny licznik dostępnych fuch po otrzymaniu wiadomosci \texttt{NEW\_JOB} od skansenu.
        Zwiększa też swój licznik odebranych wiadomości \texttt{ACK\_JOB} za każdym razem, gdy otrzyma taką wiadomość, jednocześnie zmniejszając \texttt{N}
        \item Po zakończeniu oczekiwania, krasnal zmniejsza licznik dostępnych fuch i powiadamia o tym wszystkie inne krasnale.
        W ramach tej samej wiadomości wysyła również \texttt{ACK\_JOB} do krasnali, którym wcześniej nie wysłał takich wiadomości ze względu na ich niższy priorytet.
    \end{enumerate}

    W trakcie trwania tego procesu, krasnal akceptuje wszystkie przychodzące prośby o dostanie się do portalu.

    \subsection{Przydział krasnala do portalu}

    \begin{enumerate}
        \item Krasnal, który chce ubiegać się o dostęp do portalu wysyła \texttt{REQ\_PORTAL} do wszystkich pozostałych krasnali i czeka na odpowiedzi.
        \item Krasnal kończy oczekiwanie po otrzymaniu \texttt{K-P} (\texttt{P} to liczba portali) wiadomości \texttt{ACK\_PORTAL}. Liczba \texttt{P} jest stała.
        \item W trakcie oczekiwania, krasnal zwiększa swój licznik wiadomości \texttt{ACK\_PORTAL} każdorazowo po otrzymaniu takiej wiadomości.
        \item Po zakończeniu oczekiwania, krasnal wysyła \texttt{ACK\_PORTAL} do wszystkich krasnali, którym wcześniej nie wysłał takich wiadomości ze względu na ich niższy priorytet.
        \item Następnie krasnal bierze się do roboty a po jej skończeniu ponownie ubiega się o nową fuchę.
    \end{enumerate}

    Przydział krasnala do portalu jest prostszy ze względu na fakt, że sekcja krytyczna ma stały rozmiar, który jest w pełni dostępny w momencie uruchomienia aplikacji.
    W tym przypadku również wybraliśmy specjalnie dostosowany algorytm Ricarta-Agrawali.

    Krasnal w pierwszej kolejności deklaruje chęć zajęcia portalu wysyłając \texttt{REQ} do wszystkich pozostałych krasnali.
    Pozostałe krasnale odsyłają mu \texttt{ACK} jeżeli nie ubiegają się o dostęp do portalu, mają niższy priorytet.
    Krasnal po wykonaniu fuchy i zwolnieniu portalu, wysyła \texttt{ACK} do wszystkich krasnali, którym wcześniej takiej wiadomości nie wysłał ze względu na ich niższy priorytet.

    \newpage

    \section{Pseudokod}

    \begin{algorithmic}[1]
        \State $K$ := liczba krasnali \Comment{Start}
        \State $P$ := liczba portali
        \State $S$ := liczba skansenów
        \State $ID$ := identyfikator węzła

        \State $nJobs$ := 0
        \State $clk$ := lamportInit()

        \While{True}
            \State $nAck$ := 0
            \State $ackQ$ := emptyQueue()
            \State sendReqJobToAll()
            \While{$nAck < K - nJobs$} \Comment{Fucha}
                \State $msg$, $id$ := receiveMsg()
                \If{$msg = REQ\_PORTAL$}
                    \State sendAckPortal(id)
                \ElsIf{$msg = REQ\_JOB(l)$}
                    \If{$l < clk$}
                        \State sendAckJob
                    \Else
                        \State $ackQ$.push($id$)
                    \EndIf
                \ElsIf{$msg = NEW\_JOB$}
                    \State atomic($nJobs += 1$)
                \ElsIf{$msg = ACK\_JOB(free)$}
                    \If{$free$}
                        \State $nAck$ +=1
                        \State atomic($nJobs -= 1$)
                    \Else
                        \State $nAck$ += 1
                    \EndIf
                \EndIf
            \EndWhile
            % \State \columnsep
            \State $nAck$ := 0
            \While{$nAck < K - P$} \Comment{Portal}
                \State $msg$, $id$ := receiveMsg()
                \If{$msg = REQ\_JOB$}
                    \State sendAckPortal(id)
                \ElsIf{$msg = REQ\_PORTAL(l)$}
                    \If{$l < clk$}
                        \State sendAckJob
                    \Else
                        \State $ackQ$.push($id$)
                    \EndIf
                \ElsIf{$msg = NEW\_JOB$}
                    \State atomic($nJobs += 1$)
                \ElsIf{$msg = ACK\_PORTAL$}
                    \State $nAck$ +=1
                \ElsIf{$msg = TAKE\_JOB$}
                    \State atomic($nJobs -= 1$)
                \EndIf
            \EndWhile
            \State parallel(DoTheJob, ackJobs, updateNJobs)
        \EndWhile
        \end{algorithmic}
\end{document}
